\chapter{Běžně používané sběrnice}
V oboru amatérské stavby robotů se v praxi používá několik sběrnic a protokolů pro získávání dat.
\begin{itemize}
	\item I$^{2}$C
	\item UART
	      \begin{itemize}
		      \item RS-232
		      \item RS-485
	      \end{itemize}
	\item SPI
	\item 1-Wire
	\item Nesběrnicová komunikace
\end{itemize}

\section{I$^{2}$C}
Snad nejpoužívanější je sběrnice I$^{2}$C,
také známá jako Inter-Intergrated Circuit.\footnote{ Protože je značka I$^{2}$C chráněna, používali ostatní výrobci název TWI, jedná se o prakticky stejnou sběrnici, pouze pod jiným názvem.}
Tato sběrnice byla vyvinuta firmou Philips primárně pro připojení periferií, které nevyžadovaly vysoké komunikační rychlosti.
Sběrnice podporuje jak multi-master, tak multi-slave.
Běžná rychlost je 100kbit/s, ve Fast modu je 400kbit/s. Novější revize pak umožňují až 5 Mbit/s, s touto verzí však nemusí být kompatibilní starší zařízení.
I$^{2}$C používá 7-bitovou adresu, což teoreticky znamená, že je na každé sběrnici možno provozovat až 127 zařízení, prakticky je toto číslo značně nižší.
I$^{2}$C využívá TTL.
Maximální délka sběrnice je 1 metr na 100k Baudech, sběrnice však nebyla designovaná na provoz po kabelu, jak je zvyklá ji používat většina amatérských nadšenců.\cite{nxp:UM10204}

\begin{table}[]
	\caption{Shrnutí I$^{2}$C:}
	\centering
	\begin{tabular}{|l|l|l|l|l|} \hline
		Maximální počet zařízení      & 127              \\ \hline
		Maximální délka               & 1 metr           \\ \hline
		Používanost                   & 5/5              \\ \hline
		Běžná rychlost                & 100kbit/s        \\ \hline
		Maximální teoretická rychlost & 5Mbit/s          \\ \hline
		Minimální počet vodičů        & 3(SDA, SCL, GND) \\ \hline
	\end{tabular}
\end{table}



\section{UART}
Další hojně používanou sběrnicí je UART, mezi amatérskou komunitou známého též jako sériová linka.
UART ve skutečnosti není sběrnice jako taková, jedná se spíše o něco mezi sběrnicí a protokolem.
UART definuje pouze posílaná data (0 a 1), nikoli však způsob jejich posílání či napěťové úrovně sběrnice.
O to se starají právě jednotlivé implementace/sběrnice.\footnote{I přesto se dá UART používat sám o sobě na TTL (Transistor-Transistor-Logic), v tom případě je definice napěťových úrovní ponechána jednotlivým zařízením, což může způsobit vzájemnou nekompatibilitu.}
Nejběžněji použiváné sběrnice pro UART jsou:
\subsection{RS-232} % (fold)
RS-232 je implementace UART.
Používá napěťové úrovně +15V až +5V pro logickou 1 a -5V až -15V pro logickou 0, toto platí pro vysílací část.
Přijímací část přidává dvouvoltovou histerezi kvůli rušení, což znamená +15V až +3V pro logickou 1 a -3V až -15V pro logickou 0.
RS-232 potřebuje společnou GND.\cite{RS-232} 
% subsection RS-232 (end)
\subsection{RS-485} % (fold)
RS-485 je další implementace.
Nepoužívá ale napěťové úrovně oproti společné GND, nýbrž využívá rozdílu napětí na linkách A a B, ten musí být alespoň 200mV.
To způsobuje několik věcí:
\begin{itemize}
	\item Pokud vedou obě linky podél sebe, nejlépe jsou-li kroucené, je prakticky nemožné komunikaci zarušit, což je v prostředí motorů na robotu značná výhoda.
	\item Není potřeba společná GND.
	\item Pro plný duplexní mód (jedna linka na vysílání a jedna na přijímaní) je potřeba dvojnásobný počet vodičů, tedy 4.
\end{itemize}
V závislosti na použitém převodníku z UART na RS-485 může být počet zařízení na sběrnici 32, nebo až 128.
\cite{RS-485}
% subsection RS-485 (end)

\section{SPI}
SPI, neboli Serial Peripherial Interface, je dvoukanálová synchronní multislave sběrnice.
Pro komunikaci využívá 4 vodiče, MOSI (Master Output Slave Input), neboli výstup z master a vstup do slave, MISO (Master Input Slave Output), neboli výstup ze slave a vstup do master, SCK neboli hodinový signál, a SS (Slave Select) tímto pinem nastavujeme, který slave je momentálně aktivní, tudíž na straně masteru může být počet použitých pinů větší (3 + počet slave zařírzení).
Další možná konfigurace je tzv. Daisy-chain, kdy je  MOSI masteru připojeno na MOSI prvního slave zařízení a MISO prvního slave zařízení je připojeno na MOSI dalšího slave zařízení, až MISO posledního slave zařízení je připojeno na MISO masteru.
Tato konfigurace poskytuje snížení počtu vodičů, zároveň ale snižuje i komunikační rychlost, proto informace od prvního slave zařízení musí obejít celý kruh, než se dostanou zpět k master zařízení.
Nespornou výhodou SPI je její rychlost.
Maximální rychlost není definovaná, aplikace běžně jdou až přes 10Mb/s.
Nevýhodou může být velký počet vodičů.
\cite{nxp:AN2847}

\section{1-Wire}
1-Wire je sběrnice, která jak již její název napovídá, potřebuje pouze jednu linku.
K tomu potřebuje ještě společnou GND, ale i tak to jsou pouze 2 linky, které obsáhnou napájení i komunikaci.\footnote{Sběrnici je možno provozovat i na třílinkovém módu.}
Této typologie se využívá třeba u kontaktních přístupových čipů.
Sběrnice je také poměrně známá pro svůj CRC součet, který umožňuje kontrolu odeslaných dat.
1-Wire je kompatibilní s TTL.
Každé zařízení na sběrnici má unikátní neměnnou adresu.
\cite{one-wire}

\section{Nesběrnicová komunikace}
Některé senzory nepotřebují posílat velké množství dat, a proto může být lepší nepoužít sběrnici. 
Příkladem takového senzoru může být třeba obyčejné tlačítko.
Některé senzory zmíněné v další kapitole také používájí tento způsob předávání dat.
Hlavní problém tohoto řešení je především náročnost na čas procesoru a počet vodičů.

