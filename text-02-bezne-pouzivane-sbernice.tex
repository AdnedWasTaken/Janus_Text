\chapter{Běžně používané sběrnice}
Sběrnice dávají jednotlivým modulům možnost vzájemné komunikace.
Často dochází k~záměně pojmu sběrnice a protokol, což je pochopitelné, neboť se mohu v~různých mírách překrývat.
Sběrnicí většinou rozumíme hardwarovou složku komunikace, ta definuje například počet vodičů, napěťové úrovně na těchto vodičích, atd.
Protokolem pak rozumíme formát dat posílaných po sběrnicích.
V~oboru amatérské stavby robotů se v~praxi používá několik sběrnic a protokolů pro získávání dat:
\begin{itemize}
	\item I$^{2}$C
	\item UART
	      \begin{itemize}
		      \item RS-232
		      \item RS-485
	      \end{itemize}
	\item SPI
	\item 1-Wire
	\item Nesběrnicová komunikace
\end{itemize}

\section{I$^{2}$C}
Snad nejpoužívanější sběrnice a protokol zároveň je I$^{2}$C \cite{nxp:UM10204},
také známá jako Inter-Intergrated Circuit\footnote{ Protože je značka I$^{2}$C chráněna, použili ostatní výrobci název TWI, jedná se o~prakticky stejnou sběrnici, pouze pod jiným názvem.}.
Tato sběrnice byla vyvinuta firmou Philips primárně pro připojení periferií, které nevyžadovaly vysoké komunikační rychlosti.
Sběrnice podporuje jak multi-master, tak multi-slave.
Běžná rychlost je 100~kbit/s, ve Fast modu je 400~kbit/s. Novější revize pak umožňují až 5 Mbit/s, s~touto verzí však nemusí být kompatibilní starší zařízení.
I$^{2}$C používá 7-bitovou adresu, což teoreticky znamená, že je na každé sběrnici možno provozovat až 127 zařízení, prakticky je toto číslo značně nižší.
I$^{2}$C využívá TTL.
Maximální délka sběrnice je 1 metr na 100~kBd, sběrnice však nebyla designovaná na provoz po kabelu, jak je zvyklá ji používat většina amatérských nadšenců.

\begin{table}[h]
	
	\centering
	\begin{tabular}{|l|l|l|l|l|} \hline
		Maximální počet zařízení      & 127              \\ \hline
		Maximální délka               & 1~metr           \\ \hline
		Běžná rychlost                & 100~kbit/s        \\ \hline
		Maximální teoretická rychlost & 5~Mbit/s          \\ \hline
		Minimální počet vodičů        & 3~(SDA, SCL, GND) \\ \hline
	\end{tabular}
	\caption{Shrnutí hlavních parametrů I$^{2}$C}
\end{table}



\section{UART}
Další hojně používanou sběrnicí je UART, mezi amatérskou komunitou známá též jako sériová linka.
UART ve skutečnosti není sběrnice jako taková, jedná se spíše o~něco mezi sběrnicí a protokolem.
UART definuje pouze posílaná data (0 a 1), nikoli však způsob jejich posílání či napěťové úrovně sběrnice.
O~to se starají právě jednotlivé implementace\footnote{I přesto se dá UART používat sám o~sobě na TTL (Transistor-Transistor-Logic), v~tom případě je definice napěťových úrovní ponechána jednotlivým zařízením, což může způsobit vzájemnou nekompatibilitu.}.
Nejběžněji používané implementace UARTu jsou:
\subsection{RS-232} % (fold)
RS-232 \cite{RS-232} je implementace UART.
Používá napěťové úrovně +5~V~až +15~V~pro logickou 1 a --5~V~až --15~V   pro~logickou 0, toto platí pro vysílací část.
Přijímací část přidává dvouvoltovou hysterezi kvůli rušení, což znamená +3~V~až +15~V~pro logickou 1 a --3~V~až --15~V~pro logickou 0.
RS-232 potřebuje společnou GND.
\begin{table}[h]
	
	\centering
	\begin{tabular}{|l|l|l|l|l|} \hline
		Maximální počet zařízení      & 2              \\ \hline
		Běžná rychlost                & 115200~baud/9600~baud        \\ \hline
		Maximální teoretická rychlost & 10~Mbit/s       \\ \hline
		Minimální počet vodičů        & 3~(Tx, Rx, GND) \\ \hline
	\end{tabular}
	\caption{Shrnutí hlavních parametrů RS-232}
\end{table}
% subsection RS-232 (end)
\subsection{RS-485}\label{RS-485} % (fold)
RS-485 \cite{RS-485} je další implementace UARTu.
Nepoužívá ale napěťové úrovně oproti společné GND, nýbrž využívá rozdílu napětí na linkách A~a B, ten musí být alespoň 200~mV.
To způsobuje několik věcí:
\begin{itemize}
	\item Pokud vedou obě linky podél sebe, nejlépe jsou-li kroucené, je prakticky nemožné komunikaci zarušit, což je v~prostředí motorů na robotu značná výhoda\footnote{Zároveň to výrazně zvyšuje maximální možnou délku linky, při nižších rychlostech až na 1200~m. Doporučení: rychlost [baud] * vzdálenost [m] $<$ +/- 10$^{8}$.}.  
	\item Není potřeba společná GND.
	\item Pro plný duplexní mód (jedna linka na vysílání a jedna na přijímání) je potřeba dvojnásobný počet vodičů, tedy 4.
\end{itemize}
V~závislosti na použitém převodníku z~UART na RS-485 může být počet zařízení na sběrnici 32 nebo až 128.
\begin{table}[h]
	
	\centering
	\begin{tabular}{|l|l|l|l|l|} \hline
		Maximální počet zařízení      & 32/128              \\ \hline
		Běžná rychlost                & 115200~baud/9600~baud        \\ \hline
		Minimální počet vodičů        & 2 (A, B) \\ \hline
		Maximální vzdálenost		  & 1200~m \\ \hline
	\end{tabular}
	\caption{Shrnutí hlavních parametrů RS-485}
\end{table}
% subsection RS-485 (end)

\section{SPI}
SPI \cite{nxp:AN2847}, neboli Serial Peripherial Interface, je dvoukanálová synchronní multi-slave\footnote{Na sběrnici je naráz připojen pouze jeden master (řídící jednotka) a typem sběrnice omezené množství slave (řízených) zařízení.} sběrnice.
Pro komunikaci využívá 4 vodiče:
\begin{itemize}
		\item MOSI (Master Output Slave Input) -- výstup z~master a vstup do slave
		\item MISO (Master Input Slave Output) -- výstup ze slave a vstup do master
		\item SCK -- hodinový signál
		\item SS (Slave Select) -- tímto pinem nastavujeme, který slave je momentálně aktivní
\end{itemize}
Na straně masteru může pak díky tomu být počet použitých pinů větší (3~+~počet slave zařízení).

Další možná konfigurace je tzv. Daisy-chain, kdy je  MOSI masteru připojeno na MOSI prvního slave zařízení a MISO prvního slave zařízení je připojeno na MOSI dalšího slave zařízení, až MISO posledního slave zařízení je připojeno na MISO masteru.
Tato konfigurace poskytuje snížení počtu vodičů, zároveň ale snižuje i komunikační rychlost, protože informace od prvního slave zařízení musí obejít celý kruh, než se dostanou zpět k~master zařízení.

Nespornou výhodou SPI je její rychlost.
Maximální rychlost není definovaná, aplikace běžně jdou až přes 10~Mb/s.
Nevýhodou může být velký počet vodičů, př použití standartního zapojení.


\section{1-Wire}
1-Wire \cite{one-wire} je sběrnice (a protokol zároveň), která, jak již její název napovídá, potřebuje pouze jeden vodič.
K~tomu potřebuje ještě společnou GND, ale i tak to jsou pouze 2~vodiče, které obsáhnou napájení i komunikaci\footnote{Sběrnici je možno provozovat i na třílinkovém módu.}.
Této typologie se využívá třeba u~kontaktních přístupových čipů.
Sběrnice je také poměrně známá pro svůj CRC součet, který umožňuje kontrolu odeslaných dat.
1-Wire je kompatibilní s~TTL.
Každé zařízení na sběrnici má unikátní neměnnou adresu.


\section{Nesběrnicová komunikace}
Některé senzory nepotřebují posílat velké množství dat, a proto může být lepší nepoužít sběrnici. 
Příkladem takového senzoru může být třeba obyčejné tlačítko.
Některé senzory zmíněné v~další kapitole také používájí tento způsob předávání dat.
Hlavní problém tohoto řešení je především náročnost na čas procesoru a počet vodičů.

