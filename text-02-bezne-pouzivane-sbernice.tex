\chapter{Běžně používané sběrnice}
V oboru amatérské stavby robotů se v praxi používá několik sběrnic a protokolů pro získávání dat.
\begin{itemize}
	\item I$^{2}$C 
	\item UART
		\begin{itemize}
			\item RS-232
			\item RS-485
		\end{itemize}
	\item SPI
	\item 1-Wire
	\item Nesběrnicová komunikace
\end{itemize}

\section{I$^{2}$C}
Snad nejpoužívanější je sběrnice I$^{2}$C,
také známá jako Inter-Intergrated Circuit
\footnote{ Protože je značka I$^{2}$C chráněna, používali ostatní výrobci název TWI, jedná se o prakticky stejnou sběrnici, pouze pod jiným názvem.}.
Tato sběrnice byla vyvinuta firmou Philips, primárně pro připojení periferií, které nevyžadovali vysoké komunikační rychlosti.
Sběrnice podporuje jak multi-master tak multi-slave.
Běžná rychlost je 100kbit/s, ve Fast modu je 400kbit/s. Novější revize pak umožňují až 5 Mbit/s, s touto verzí však nemusí být kompatibilní starší zařízení.
I$^{2}$C používa 7-bitovou adresu, což teoreticky znamená, že je na každou sběrnici možno provozovat až 127 zařízení, prakticky je toto číslo značně nižší.
Maximální délka sběrnice je 1 metr na 100k Baudech, sběrnice však nebyla designovaná na provozu po kabelu, jak ji používá většina amatérských nadšenců.\cite{https://www.nxp.com/docs/en/user-guide/UM10204.pdf}

\begin{table}[]
	\caption{Shrnutí:}
	\centering
	\begin{tabular}{|l|l|l|l|l|} \hline
		Maximální počet zařízení & 127   \\ \hline
		Maximální délka & 1 metr  \\ \hline
		Používanost & 5/5   \\ \hline
		Běžná rychlost & 100kbit/s \\ \hline
		Maximální teoretická rychlost &  5Mbit/s  \\ \hline
		Minimální počet vodičů & 3(SDA, SCL, GND) \\ \hline
	\end{tabular}
\end{table}



\section{UART}
Další hojně používanou sběrnicí je UART, mezi amatérskou komunitou znám též jako sériová linka.
UART ve skutečnosti není sběrnice jako taková, jedná se spíše o něco mezi sběrnicí a protokolem.
Nejběžněji použiváné sběrnice pro UART jsou:
\subsection{RS-232} % (fold)
\label{sub:RS-232}
	RS-232 je implementace UART
% subsection RS-232 (end)
\subsection{RS-485} % (fold)
\label{sub:RS-485}

% subsection RS-485 (end)

% \section{Musíme si dokonale promyslet obsah}
% Jakmile víme, co chceme říci a komu, musíme si rozvrhnout látku.
% Ideální je takové rozvržení, které tvoří logicky přesný a psychologicky stravitelný celek, ve kterém je pro všechno místo a jehož jednotlivé části do sebe přesně zapadají.
% Jsou jasné všechny souvislosti a je zřejmé, co kam patří.

% Abychom tohoto cíle dosáhli, musíme pečlivě organizovat látku.
% Rozhodneme, co budou hlavní kapitoly, co podkapitoly a jaké jsou mezi nimi vztahy.
% Diagramem takové organizace je graf, který je velmi podobný stromu, ale ne řetězci.
% Při organizaci látky je stejně důležitá otázka, co do osnovy zahrnout, jako otázka, co z~ní vypustit.
% Příliš mnoho podrobností může čtenáře právě tak odradit jako žádné detaily.

% \section{Musíme začít psát strukturovaně}
% Máme-li tedy myšlenku, představu o~budoucím čtenáři, cíl a osnovu textu, můžeme začít psát.
% Při psaní prvního konceptu se snažíme zaznamenat všechny své myšlenky a názory vztahující se k~jednotlivým kapitolám a podkapitolám.
% Každou myšlenku musíme vysvětlit, popsat a prokázat.
% Hlavní myšlenku má vždy vyjadřovat hlavní věta a nikoliv věta vedlejší.
