\newpage
\chapter*{Závěr}
\addcontentsline{toc}{chapter}{Závěr}

Ke dni odevzdání je funční Omni-ultra, všesměrový ultrazvukový senzor a Janus protocol jako takový.
Ve vývoji se nachází knihovna pro práci s~jednotlivími senzory.
Začátečníkům z~robotických kroužků na SPŠ Sokolské Brno\cite{sokolska} a pobočce DDM Helceletova Brno -- Robotárně\cite{robotarna} se tak do rukou dostává, zatím omezené množství senzorů, se kterými mohou experimentovat.
Pokročilejší uživatelé pak mají možnost požít ekosystém pro vytvoření vlastních senzorů a modulů.
Všechny mnou vytvořené součásti ekosystému jsou volně dostupné jako Open Source, na Githubu (viz \ref{Literatura}).

I~přesto asi nedjdůležitě výsledek, který si z~práce odnáším já osobně jsou zkušenosti s~většími komplexními projekty, v~porovnání s~mými dosavadními získané hlavně prací na samostatných malých projektech.

Janus protocol a hlavně jednotky, které na něm operují, se mají stále kam rozvíjet a to jak kvalitativně, tak i kvantitativně.

Do budoucna mám v~plánu dodělat komplení a jednoduše srozumitelnou dokumentaci jak pro celý protokol, tak pro jednotlivé senzory.
Především kvůli tomu, aby si již zkušenější uživatelé mohli přidávat sami další senzory do ekosystému a poskytovali tím méně zkušeným uživatelům dostupnou sadu senzorů pro vlastní experimentování.
Pro pohodlnost práce plánuji přidat modul pro Sigrok(program pro analýzu dat z~logického analyzéru), určený k~rychlé kontrole dat na sběrnici.
