\newpage
\chapter*{Závěr}
\addcontentsline{toc}{section}{Závěr}

Ke dni odevzdání je funční Omni-ultra, všesměrový ultrazvukový senzor a Janus protocol jako takový.
Ve vývoji se nachází knihovna pro práci s jednotlivími senzory.
Začátečníkům z robotických kroužků na SPŠ Sokolské Brno a pobočce DDM Helceletova Brno -- Robotárně se tak do rukou dostává, zatím omezené množství senzorů, se kterými mohou experimentovat.

Janus protocol a hlavně jednotky, které na něm operují, se mají stále kam rozvíjet a to jak kvalitativně, tak i kvantitativně.

Do budoucna mám v plánu dodělat komplení a jednoduše srozumitelnou dokumentaci jak pro celý protokol, tak pro jednotlivé senzory.
Především kvůli tomu, aby si již zkušenější uživatelé mohli přidávat sami další senzory do ekosystému a poskytovali tím méně zkušeným uživatelům dostupnou sadu senzorů pro vlastní experimentování.
Pro usnadnění práce plánuji přidat modul pro Sigrok (program pro analýzu dat z logického analyzéru), určený k rychlé kontrole dat na sběrnici.
