\chapter{Slovo Jarka}
Trochu bych nesouhlasil s~Romanem ohledně „hlavního výstupu vaší roční činnosti“ a také se zdroji, odkud je vhodné čerpat inspiraci.

Rád bych tato dvě témata na závěr trochu rozebral a zároveň přidal pár slov o~prezentacích, které by měly tvořit podstatnou část vaší práce.

\section{Hlavní výstup vaší roční činnosti}
U~některých oborů možná platí, že hlavním výstupem vaší roční činnosti nejsou data nebo zařízení, nýbrž právě odborný text.
Ovšem z~vlastní zkušenosti mohu říct, že pokud předvedete funkční výtvor (a to ať už softwarový balík pro vývoj a řízení aplikací s~mikročipy, výukový webový portál, univerzální ovládací pult, nebo regulovatelný napájecí zdroj) budete mít na 90 \% větší úspěch než čistě teoretická práce.

Samozřejmě pokud někdo vyvrátí teorii relativity nebo vymyslí novou a lepší periodickou tabulku prvků, bude mít pravděpodobně lepší pozici než vy.
Proto ale musí být vaše práce co možná nejlepší.

Pokud donesete výrobek, který je inovativní, nadčasový, velmi nápaditý a případně vyrobitelný nebo dokonce komerčně prodatelný, a dokážete ho při prezentaci prodat (o~důležitosti prezentace více informací níže), většina porotců vám promine i formální nedostatky a krátký rozsah práce, protože jste jim to předvedli naživo (minimálně toto platí v~rámci oboru strojírenství, elektra a informatiky a podle mě i fyziky, učebních pomůcek atd.)

Kdybych to vzal do extrému, tak práce, která nemá text, ale je velmi zajímavá pro svůj výrobek (zařízení), může klidně vyhrát celostátní kolo SOČ.
Ovšem když přijdu s~textem, kde tento výrobek dokonale popisuji, ale nedovezu, nepředvedu, neukáži, že je funkční, tak jsem na tom hůř než v~prvním případě.

Toto jsou ovšem specifika spíše techničtějších oborů (s~kterými mám zkušenost) a je možné, že v~přírodních vědách (jako chemie, matika, biologie) má spíše Roman pravdu, ale nejsem si tím úplně jistý.
Zvažte sami~:-)

\section{Kde čerpat inspiraci}
Roman jako inspiraci doporučoval theses.cz, ovšem já bych vás spíše odkázal na archiv SOČ (\url{http://soc.nidm.cz/archiv}) a to ze tří důvodů:

\begin{enumerate}[label=\alph*)]
	\item Uvidíte styl a způsob zpracování úspěšných prací SOČ, které vytvořili studenti ve vašem věku a které se porotcům líbily.
	\item Nemusíte se probírat stovkami tisíc závěrečných prací, ale jednoduše si vyberete váš obor a projdete několik nejlepších prací za posledních pár let.
	\item Styl bakalářských a diplomových prací se od SOČek trochu liší a občas je lepší se držet zaběhnutých pravidel SOČ.
\end{enumerate}
Samozřejmě si můžete projít i několik vysokoškolských prací a třeba v~nich najdete i lepší inspiraci.

Jinak nad samostatným formátováním (či některými detaily) neztrácejte mnoho času, protože vám pravděpodobně schází ještě podstatnější věci.
A~také platí, že co porotce/obor, to jiný názor na některé detaily formátování SOČek :-)

\section{Prezentace}
Sebelepší práce bez dobré prezentace je prakticky k~ničemu.
Porotci v~nižších kolech (okresních i krajských) nemají často čas prostudovat si text práce předem.
Tudíž jej občas vidí poprvé, až když danou práci prezentujete.

Pokud budete mít dobrou prezentaci, ve které svoji práci dobře prodáte – ukážete, co jste dělali, jak jste to dělali, co je vaše práce, ale hlavně co je tak unikátního na vaší práci a proč zrovna vy byste měli vyhrát), tak máte z~poloviny vyhráno.
Prezentace tvoří klidně i polovinu hodnocení vaší práce.
Nebo víc.

Proto je podle mě důležité věnovat minimálně stejné množství času přípravě prezentace jako textu.
Pokud postoupíte do dalšího kola, máte většinou možnost si svou práci vzít a do týdne ji upravit/doladit.
Samozřejmě že pokud budete mít perfektní text hned do prvního kola SOČ, budete mít výhodu vůči ostatním a lepší startovací pozici do dalších kol, ale v~případě nedostatku času (což většinou bývá) je vhodnější rozdělit si čas mezi tvorbu textů, prezentace a případně samostatného výrobku.

Samozřejmě se nebojte inspirovat u~svých kolegů z~minulých let například na serveru YouTuBe (hledejte pod „CP SOČ 2013“ a „SOČ 2013 – Brno – krajské kolo“).

O~prezentacích se toho dá samozřejmě napsat mnoho, ale to si necháme třeba na příště ;-)

\vspace{\baselineskip}
\noindent SOČce zdar!

\vspace{\baselineskip}
\noindent \B{Jarek Páral}\\
\url{paral.jarek@gmail.com}