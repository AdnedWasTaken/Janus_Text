\chapter*{Úvod}
\addcontentsline{toc}{chapter}{Úvod} % přidá položku úvod do obsahu
%odsazení od vrchu moc velké
V~poslední době se robotické soutěže těší stále většímu zájmu jak veřejnosti, tak konstruktérů.
Mezi v~česku známé robotické soutěže patří například Pražský robotický den\cite{Prague-robotic-day} a Robotiáda\cite{Robotiada}.
Senzory dosahují různé kvality a používají různé komunikační protokoly a sběrnice, což může být problémem pro začínající konstruktéry, kteří se díky nejednotné nabídce musí starat o~věci jako duplicitní adresy na sběrnici, kolize dvou knihoven řídících jednu sběrnici a hardwarové problémy dané sběrnice.

Cílem této práce je těmto začínajícím konstruktérům poskytnou nástroj, který většinu problémů se senzorikou vyřeší za ně, pro pokročilejší konstruktéry pak nabízí úsporu času.
Tito konstruktéři nemusí již senzory, které se neustále opakují, stavět, zapojovat a programovat vždy znovu, ale dostanou do rukou téměř plug-n-play řešení.


\newpage