\chapter*{Úvod}
\addcontentsline{toc}{chapter}{Úvod} % přidá položku úvod do obsahu
V poslední době se robotické soutěže těší čím dále vetšímu zájmu jak veřejnosti, tak konstruktérů. Senzory dosahují různé kvality a používají různé komunikační protokoly a sběrnice, což může být problém pro začínající konstruktéry, kteří se kvůli tomuto zmatku starat o věci jako duplicitní adresy na sběrnici, kolizi dvou knihoven řídících jednu sběrnici a hardwarové problémy dané sběrnice. Cílem této práce je těmto začínajícím konstruktérům poskytnou nástroj, který většinu problémů se senzorikou vyřeší za ně, pro pokročilejší konstruktéry pak nabízí úsporu času, nemusí již senzory, které se neustále opakují stavět, zapojovat a programovat pokaždé znova, ale dostanou do rukou téměř plug-n-play řešení.

%Účelem je stručně a věcně seznámit se záměrem a řešením tématu, důvodem jeho volby, stručně nastínit problém, který má být řešen (a proč); cíl práce (co je předmětem řešení, jakým způsobem se postupuje a k~čemu se má dospět -- co bude výsledkem; cíl práce je „páteří“ práce, je jedním z~kritérií i pro posuzování řešení tématu, zpracování práce); dále v~úvodu uvést, jaký je postup řešení, metody; výzkumná otázka a ústřední hypotéza (obvykle u~empirických výzkumů) -- vše formou charakteristiky záměru tématu a postupu jeho řešení (zdůvodnění), rozpracování je pak v~textu práce.
\newpage