\chapter{Běžně používané senzory na soutežních robotech}

Každý robot potřebuje mít způsob interakce s okolím.
Tuto interakci zajišťují právě senzory.
Naprostá většina amatérských týmů nemá prostor ani prostředky vytvářet vlastní senzory.
Používání průmyslových senzorů pak zněmožňuje několik faktorů.
Asi nejdůležitějším je cena, dále pak jejich velikost a hmotnost způsobená jejich robustností a kvalitou provedení.
Týmy jsou tedy nuceni používat hotové destičky obsahující pro ně často obskurní komponenty.

Senzory se dají rozdělit do několika kategorií:
\begin{itemize}
    \item senzory vzdálenosti
        \begin{itemize}
            \item ultrazvukové senzory
            \item IR senzory
            \item Lidar/Radar/Sonar
        \end{itemize}
    \item senzory barvy
        \begin{itemize}
            \item black-white senzory
            \item RGB senzory
        \end{itemize}
    \item akcelerometry
    \item gyroskopy
    \item enkodéry
    \item kompasy
    \item komplexní polohové senzory
\end{itemize}

\section{Senzory vzdálenosti}

Senzory pro zjišťování vzdálenosti dodávájí robotovy poměrně primitivním způsobem schopnost přibližně určit svou polohu.
Podmínky pro jejich použití jsou však často velmi specifické a nedají se pro to sami o sobě použít pro přesnější lokalizaci.

\subsection{Ultrazvukové senzory}

Asi nejpoužívanější senzory vzdálenosti jsou senzory ultrazvukové. ty fungují na principu vyslání ultrazvuového pulzu a čekání na jeho návrat.


Nejběžněji používaný z nich je HC-SR04. 
Ten obsahuje ultrazvukový přijímač a vysílač, spolu s dodatečnou elektronikou.
má čtyři vývody: 
\begin{table}[]
	\caption{Vývody HC-SR04:}
	\centering
	\begin{tabular}{|l|l|l|l|l|} \hline
		GND & společná zem/-   \\ \hline
		VCC/5V & napájení 5V/+  \\ \hline
		Echo & Návrat měřené vzdálenosti   \\ \hline
		Trig & Spouštění meření \\ \hline
	\end{tabular}
\end{table}
Měření započne posláním logické 1 na pin Trig po dobu alespoň 10$\mu$S.
Po té co Trig opět přepneme na logickou 0, vyšle senzor 8 40-ti kHz pulzů, zárověň nastaví na pinu Echo logickou 1.
Po přijmutí odraženého ultrazvukového signálu je na pinu Echo opět nastavena logická 0.
Mikrokontroleru poté pouze zbýva měřit jak dlouho byla na pinu Echo logická 1, tento čas pak dosadí do rovnice:
\begin{center}
    vzdálenost=(naměřený čas*rychlost zvuku)/2
\end{center} 
Senzor je schopen měřit vzdálenosti od 2cm do 4m.
Senzor má nevalnou přesnost.
Největší úskalí při používaní tohoto senzoru nastává, pokud je na hřišti více robotů/nesynchronyzovaných senzorů, kdy se senzory mohou navzájem rušit.
Další problém může vyvstat při používaní pouze jednoho mikroprocesoru a několika ultrazvukových senzorů, kdy čas potřebný na změření všech senzorů přesáhne únosnou mez, čímčž zásadně prodlouží reakční čas robota.
Tento problém řeší použití sekundárního procesoru pro obsloužení měření.