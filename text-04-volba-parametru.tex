\chapter{Volba parametrů}
\section{Sběrnice}
Pro svou práci jsem zvolil sběrnici RS-485.
Hlavní důvody pro toto rozhodnutí jsou:
\begin{itemize}
    \item Nezarušitelnost
    \item Počet zařízení
    \item Počet vodičů
    \item Celková robustnost
\end{itemize}

\section{Protokol}
Na sběrnici RS-485 se nabízela možnost použít protokol Modbus, což je průmyslový standard.
Byla to původně i moje volba, ale od jeho použití mě odradila jeho komplexnost.
Rozhodl jsem se proto vytvořit vlastní protokol, pracovně pojmenovaný Janus protocol.

\section{Zpracované senzory}
Senzory plánované pro implementaci do této sady zahrnují:
\begin{itemize}
    \item Všesměrový ultrazvukový senzor
    \item Sběrač\footnote{Viz kapitola \ref{Sberac}} pro ultrazvukové senzory HC-SR04 
    \item Senzor pro Line follower
    \item RGB senzor
    
\end{itemize}
Kromě senzorů mají být součástí sady i další jednotky:
\begin{itemize}
    \item Battery pack -- inteligentní baterie
    \item Co-processing unit -- přídavná výpočetní jednotka
    \item Rozbočovač -- v případě potřeby možnost rozšířit sběrnici o více zařízení
    \item Buffer -- Kombinace co-processing unit a rozbočovače
\end{itemize}
\section{Procesory}

Rozhodl jsem se použít pro všechny senzory i řídící jednotku procesory z řady ESP32.
Některé z důvodů:
\begin{itemize}
    \item Build-in podpora pro RS-485 přímo v UART driveru ESP32
    \item Kompletní podpora C++v14 a STL na rozdíl od běžně používaných ATMEGA čipů
    \item Kompatibilita se spoustou rozhraní a sběrnic -- může sloužit jako převodník mezi ostatními sběrnicemi a Janus protocol
    \item OTA -- možnost nahrávání programu přes internet\footnote{Snaha o co nejjednodušší dodatečné opravy systému, kdy si senzory budou samy schopny dostahovat nejnovější firmware, a na obalu nebudou muset být vyvedené programovací konektory.}
    \item WiFi a Bluetooth -- v případě nutnosti bezdrátové komunikace
    \item Kompatibilita s Arduino\footnote{Přestože samotný kód protokolu nevyžaduje Arduino a je postaven čistě na ESP-IDF, je Arduino prostředí pro mnoho začátečníků přijemnější pro práci než většina ostatních. Jeho jednoduchost a kompatibilita s jinými čipy však může zásadně limitovat jeho funkce, proto není obsluha protokolu napsaná právě v něm.}
\end{itemize}